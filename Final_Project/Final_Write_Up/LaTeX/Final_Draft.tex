\documentclass[article,12pt]{article}

\usepackage[utf8]{inputenc}
\usepackage{amssymb, amsfonts,amsthm}
\usepackage[fleqn]{amsmath} % Math packages
\numberwithin{equation}{section}
\usepackage{listings}
\usepackage[top=1in, bottom=1in, left=1in, right=1in]{geometry}
%\usepackage[T1]{fontenc} % Use 8-bit encoding that has 256 glyphs
%\usepackage{fourier} % Use the Adobe Utopia font for the document - comment this line to return to the LaTeX default
\usepackage[english]{babel} % English language/hyphenation
\usepackage{enumerate}
\usepackage[usenames,dvipsnames]{color} % Required for custom colors
\usepackage{listings} % Required for insertion of code
\usepackage{courier} % Required for the courier font
\usepackage{tikz} 
\usepackage{sectsty}
\usepackage{multicol} % Required for multiple columns
%\usepackage{tabu} % Option for Table Construction
\usepackage{epigraph} 
\setlength{\epigraphwidth}{\textwidth}
\usepackage{hologo}
\usepackage[font=small,labelfont=bf]{caption} % Specifying Captions
\usepackage{multirow} % TAbles
\usepackage{changepage} % Change margins



\usepackage{blindtext}
\usepackage{setspace} % Spacing
\usepackage{csquotes}% Recommended
\usepackage{psvectorian} % Cool ornaments

\usepackage{hyperref} % hyper links
\hypersetup{
	colorlinks=true,
	linkcolor=blue,
	filecolor=magenta,      
	urlcolor=cyan,
	pdftitle={Prospectus},
	pdfpagemode=FullScreen,
	citecolor= black
}

\urlstyle{same}


\renewcommand{\baselinestretch}{1.5} % Spacing

\usepackage[style = numeric, sorting = none, backend=biber]{biblatex}
\addbibresource{references.bib}

\usepackage{graphicx}
\graphicspath{{figs/}} %Setting the graphicspath


\begin{document}

\begin{center}
Lab Report

Title: Measuring the Spatio-Temporal Distribution of Environmental Hazards Across Minneapolis - First Draft\\
Notice: Dr. Bryan Runck\\
Author: Rob Hendrickson\\
Date: 11/09/2022\\~\\

Project Repository: \url{https://github.com/RwHendrickson/GIS5571/blob/main/Final_Project}\\
Google Drive Link: NA\\
Time Spent: 20 hours

\end{center}

\section*{Abstract}
It is understood that some parts of Minneapolis experience a greater burden of environmental hazard than others. Anecdotally and visually, this can be correlated to restrictive housing practices in the early to mid 20th century. This project aims to quantify the cumulative environmental harm across Minneapolis at a fine spatial resolution with the intention of spatially correlating this index with modern demographics and restrictive housing practices.


\section*{Problem Statement}

How do we spatially measure environmental harm? There are a plethora of factors involved in measuring environmental risk. This project will focus on air quality.

%, however, soil samples from the local initiative, \href{https://www.facebook.com/groups/425283655203308/}{Edible Boulevards}, and \href{http://doi.org/10.13020/D6C016}{tree canopy} are other avenues of knowledge. 
%
%Some potential variables involved in determining environmental risk may include: particulate matter 2.5 (PM2.5 – $\frac{micrograms}{meter^3}$), volatile organic compounds (VOCs - $\frac{micrograms}{meter^3}$), SO4, NO, benzene, lead concentration in soil, Annual Average Daily Traffic (AADT), health metrics, and tree coverage. 

Three indicators of air pollution were considered as factors in determining environmental risk. Current Annual Average Daily Traffic ($T_{2022}$), current industrial emissions ($I_{2020}$), as well as cummulative industrial emissions from the past 14 years ($\Sigma I = \sum_{y=2006}^{2020} I_{y}$). The industrial emissions are further categorized by pollutant. There are many. This project focuses on some of the primary modern concerns, namely particulate matter 2.5 (PM2.5), PM10, PM, Sulfur Dioxide (SO2),  and volatile organic compounds (VOCs). See methods section for further notation.

These variables are rasterized and then dispersed throughout an 8km buffered Minneapolis into a 50 meter resolution raster using a Gaussian Kernel convolution. Validating these models is my next step. There's also this idea I've had of looking deeper into dispersion modeling.

The resulting distribution of raster values were explored. All $I_{2020}$ and $\Sigma I$ were logarithmically transformed (maintaining zero values) then each raster layer was normalized by their maximum value. Weighted linear combinations of these layers were summed up to create some initial environmental hazard indices. These weights could be further refined by consulting with community members, which I am working on after model validation.

I think the final step (after getting a good model and polling stakeholders for layer weights) is to perform some zonal statitistics of these indices using areas that were redlined by the Home Owners’ Loan Corporation in the 1930’s (HOLC areas), homes with \href{https://pressbooks.umn.edu/mappingprejudicecurriculum/chapter/what-is-a-racial-covenant/}{racially-restrictive deeds} (MP areas), and census block groups. Though maybe I should reclassify the indices first? For the HOLC and MP areas pearson correlation may be explored or some sort of confusion matrix!?

Spatial weights could be explored in the census block groups. The project may end with some spatial regression with these zonal statistics and demographics such as race/ethnicity, median income, and immigration status, however, this is probably going to be next semester. 

%It is understood that some parts of Minneapolis experience a greater burden of environmental hazard than others. Anecdotally and visually, this can be correlated with areas that were redlined by the Home Owners’ Loan Corporation (HOLC) in the 1930’s. Ultimately, these disparities can be traced back to the \href{https://pressbooks.umn.edu/mappingprejudicecurriculum/chapter/what-is-a-racial-covenant/}{racially-restrictive deeds} that were authored in Minnesota from 1910 to 1953 and are an example of how historic racism affects the lives of people today. This project aims to measure that disparity with the intention of spatially correlating it with demographics and restrictive housing practices.

%\fbox{ % Workflow
%	\begin{minipage}{.8\linewidth}
%		\begin{center}
%			\begin{minipage}{\linewidth}
%				\includegraphics[width=.5\linewidth,angle=270]{steps}
%			\end{minipage}
%			\captionof{figure}{Project Steps}
%		\end{center}
%	\end{minipage}
%}\\
{
	\scriptsize
	\begin{tabular}{|l|p{.2\linewidth}|p{.2\linewidth}|p{.2\linewidth}|p{.1\linewidth}|p{.1\linewidth}|p{.1\linewidth}|}
	\hline	& \textbf{Requirement} & \textbf{Defined As} & \textbf{(Spatial) Data} & \textbf{Attribute Data} & \textbf{Dataset} & \textbf{Preparation} \\ \hline
		1 & Model Traffic Emissions Dispersion         &                                                                & MnDoT’s Current AADT Segments           &                                                   & \href{https://gisdata.mn.gov/dataset/trans-aadt-traffic-segments}{MnDOT}                                                                                                                 &             \\ \hline
		2 & Model Industrial Emissions Dispersion      &                                                                & MPCA’s Permitted Facility Air Emissions &                                                   & \href{https://www.pca.state.mn.us/air/permitted-facility-air-emissions-data}{MPCA}                                                                                                      &             \\ \hline
		3 & Validate Models                            & Check model output with Observed data                          & PurpleAir                               & Pm2.5, VOCs ($\frac{\mu g}{m^3}$) & \href{https://api.purpleair.com/}{PurpleAir}                                                                                                                                                 &             \\ \hline
		4 & Synthesize Models                            & Ensure that all environmental risk variables are in normal-ish distributions and range from 0 to 1 &                  &                                                   &                                                                                                                                            &             \\ \hline
		5 & Create environmental cost surface & Experiment with different weights for the variables                &                                         &                                                   &                                                                                                                                                                                                         &             \\ \hline
		6 & Explore Results                       & Explore the spatial distribution                       & HOLC, Mapping Prejudice, Census         &                                                   & \href{https://gisdata.mn.gov/dataset/us-mn-state-metc-plan-historic-holc-appraisal}{HOLC} \newline  \href{https://mappingprejudice.umn.edu/racial-covenants/maps-data}{Mapping Prejudice} \newline \href{https://data2.nhgis.org/main}{Census} &            
		\\ \hline
		7 & Correlation Analysis                       & Explore how variables/indices correlate                        & HOLC, Mapping Prejudice, Census         &                                                   & \href{https://gisdata.mn.gov/dataset/us-mn-state-metc-plan-historic-holc-appraisal}{HOLC} \newline  \href{https://mappingprejudice.umn.edu/racial-covenants/maps-data}{Mapping Prejudice} \newline \href{https://data2.nhgis.org/main}{Census} &              \\ \hline
		8 & Spatio-Temporal Modeling                   &                                                                &                                         &                                                   &                                                                                                                                                                                                         &       \\ \hline     
	\end{tabular}
\captionof{table}{Project Steps}}

\section*{Input Data}
{
	\scriptsize
	\begin{tabular}{|l|p{.2\linewidth}|p{.2\linewidth}|p{.4\linewidth}|}
	& \textbf{Title}                              & \textbf{Purpose in Analysis}     & \textbf{Link to Source}    
	\\
	1 & Metropolitan Council's 2010 County, City, \& Township Boundaries of the Twin Cities \cite{metrocouncil2010}     & Defining extent & \url{https://gisdata.mn.gov/dataset/us-mn-state-metc-bdry-census2010counties-ctus}                                                                 \\
	2 & MnDoT’s Current AADT Segments \cite{mndot_reg}     & Modeling and Risk Index & \url{https://gisdata.mn.gov/dataset/trans-aadt-traffic-segments}                   \\
	3 & MPCA’s Permitted Facility Emission \cite{mpca_emitter} & Modeling and Risk Index & \url{https://www.pca.state.mn.us/air/permitted-facility-air-emissions-data}        \\
	4 & PurpleAir Observed Air Quality    & Validating Model        & \url{https://api.purpleair.com/}                                                   \\
%	4 & Tree Canopy \cite{tree2015}                        & Risk Index              & \url{http://doi.org/10.13020/D6C016}                                               \\
%	5 & Soil Quality                       & Risk Index              &                                                                              \\
	5 & HOLC Grades                        & Correlation             & \url{https://gisdata.mn.gov/dataset/us-mn-state-metc-plan-historic-holc-appraisal} \\
	6 & Restrictive Deeds                  & Correlation             & \url{https://mappingprejudice.umn.edu/racial-covenants/maps-data}                \\
	7 & Demographics \cite{ipums}                       & Correlation             & \url{https://data2.nhgis.org/main}                                               
\end{tabular}
\captionof{table}{Data Sources}}

\section*{Methods}
The following pages are a rough sketch of the methodology thus far. My Github Repository is all up to date, I just haven't had time to formally write up everything that's been done.  
\newpage~\newpage~\newpage~\newpage
%To be determined... But for modeling air quality I'm considering using resources from both \href{https://www.pca.state.mn.us/business-with-us/air-quality-modeling}{MPCA}  and \href{https://plumepgh.org/model_data.html}{Plume Pittsburg}. I think the model validation will involve some RMSE, residual, and Pearson Correlation calculations. For the correlation analysis, I was thinking of doing something similar to an earlier project of mine using SLX, SLY, Durbin, and different GWR spatial regressions, but I'm open to suggestions! MPCA also has an \href{https://www.pca.state.mn.us/business-with-us/air-emissions-risk-analysis-aera}{air emissions risk assessment} that I will explore further when considering the risk index.

\section*{Results}
\begin{adjustwidth}{-.55in}{-.55in}	

	\fbox{\begin{minipage}{\linewidth} 
			\begin{minipage}{.5\linewidth}
				\includegraphics[width=\linewidth]{0PM-100T_index}
			\end{minipage}
			\begin{minipage}{.5\linewidth}
				\includegraphics[width=\linewidth]{100PM-0T_index}
			\end{minipage}
			\captionof{figure}{Visualizations of Normalized $T_{2022}$ (Left) and Transformed \& Normalized $I_{2020}$ PM2.5 emissions (Right).}
	\end{minipage}}
	
\end{adjustwidth}

\newpage
\begin{adjustwidth}{-.55in}{-.55in}	
	\fbox{ 
		\begin{minipage}{\linewidth}
			\begin{center}
				\begin{minipage}{\linewidth}
					\includegraphics[width=\linewidth]{50PM-50T_index.png}
				\end{minipage}
				\captionof{figure}{An example of an environmental hazard cost surface (50\% Current Traffic, 50\% Current PM2.5 Emissions).}
			\end{center}
		\end{minipage}
		
	}
\end{adjustwidth}
%
%\begin{adjustwidth}{-.55in}{-.55in}
%	\begin{center}
%		\fbox{
%			\begin{minipage}{\linewidth}
%				\captionof{figure}{Geographically weighted regressions (Air Quality). Fixed 2 kilometer bandwidth.}
%				\begin{center}	
%					\fbox{ % Figure - Benzene GWR - All
%						\begin{minipage}{.5\linewidth}
%							\begin{center}
%								\begin{minipage}{\linewidth}
%									\includegraphics[width=\linewidth]{Benzene GWR - All Variables}
%								\end{minipage}
%								\captionof*{figure}{Regressand: $A_B$, Regressor: $T$ and $R_{NW}$.}
%							\end{center}
%						\end{minipage}
%					}
%				\end{center}
%				
%				\begin{center}	
%					\fbox{ % Figure - Naphthalene GWR - All
%						\begin{minipage}{.5\linewidth}
%							\begin{center}
%								\begin{minipage}{\linewidth}
%									\includegraphics[width=\linewidth]{Naphthalene GWR - All Variables}
%								\end{minipage}
%								\captionof*{figure}{Regressand: $A_N$, Regressor: $T$ and $R_{NW}$.}
%							\end{center}
%						\end{minipage}
%					}
%				\end{center}
%		\end{minipage}}	
%		
%	\end{center}
%\end{adjustwidth}
%
%\begin{tabular}{lrrrrr}
%	\hline
%	{} &     $R^2$ &  Pearson Coeff &     p-value &  Residual Morans I &     p-value \\
%	\hline
%	NonSpatial &  0.0888 &         0.2980 &  3.7698e-09 &             0.4336 &  1.1466e-18 \\
%	SLX        &  0.1041 &         0.3226 &  1.4926e-10 &             0.4393 &  3.5773e-19 \\
%	SLY        &  0.2682 &         0.5179 &  3.4336e-27 &            -0.0877 &  8.9419e-02 \\
%	Durbin     &  0.2673 &         0.5170 &  4.3190e-27 &            -0.0774 &  1.3412e-01 \\
%	GWR        &  0.4777 &         0.6959 &  9.7322e-56 &            -0.4243 &  7.2401e-18 \\
%	\hline
%\end{tabular}
%\captionof{table}{Accuracy checks of regressions. Regressand: $T$, Regressor: $R_{NW}$.}
%\vspace{.2in}
%\begin{adjustwidth}{-.55in}{}
%	\begin{tabular}{lrrrrllll}
%		\hline
%		{} &  CONSTANT &       p-value &  $R_{NW}$ &      p-value & $R_{NW}$\_lag &  p-value & $T$\_lag & p-value \\
%		\hline
%		NonSpatial &  33.44931 &  1.98079e-135 &      -0.12088 &  3.76979e-09 &                - &        - &              - &       - \\
%		SLX        &  34.99942 &  8.84001e-115 &      -0.05236 &  1.21008e-01 &         -0.11258 &  0.01219 &              - &       - \\
%		SLY        &  12.48954 &   5.28596e-07 &      -0.05557 &  3.39681e-03 &                - &        - &        0.63771 &     0.0 \\
%		Durbin     &  13.00777 &   4.13289e-06 &      -0.04503 &  1.39290e-01 &         -0.01866 &  0.65725 &        0.62976 &     0.0 \\
%		\hline
%	\end{tabular}
%	\captionof{table}{Coefficients of regression models. Regressand: $T$, Regressor: $R_{NW}$.}
%	
%\end{adjustwidth}



\section*{Results Verification}

The air quality modeling will be verified with the observations from PurpleAir Sensors. Literature will also be consulted on how best to model air pollutant dispersion.

\section*{Discussion and Conclusion}

	Environmental justice (EJ) continues to expand and incorporate different conceptions of space, environment, and justice. Contemporary EJ writers often cite a need for more community outreach, education, and inclusive decision-making as well as innovative collaboration between planners, policy-makers, academics, and citizens to achieve profound environmental justice \cite{walker2010, corburn2003, pearsall2010}. \textcite{pearsall2010} emphasize a need for well defined indicators that are spatially focused, not aggregate measures, to gauge a policy's success at implementing environmental justice. This new workflow and index are a couple more tools we can use to collectively reckon with our history, assess our current situation, and work toward reparations. 

\begingroup           % Ctrl T to uncomment
\setstretch{1}
\setlength\bibitemsep{12pt}  % length between two different entries
\printbibliography
\endgroup

\section*{Self Score}
\setstretch{.2}
\begin{tabular}{|p{.2\linewidth}|p{.2\linewidth}|p{.2\linewidth}|p{.1\linewidth}|}
	\hline
	\textbf{Category}            & \textbf{Description}                                                                                                                                                                                                                                                                                                                                              & \textbf{Points Possible} & \textbf{Score} \\ \hline
\vspace{.2in}\textbf{Structural Elements} & {\tiny All elements of a lab report are included (2 points each): Title, Notice: Dr. Bryan Runck, Author, Project Repository, Date, Abstract, Problem Statement, Input Data w/ tables, Methods w/ Data, Flow Diagrams, Results, Results Verification, Discussion and Conclusion, References in common format, Self-score}                                        & \vspace{.2in}28              &   \vspace{.2in}20    \\ \hline
	\vspace{.2in}\textbf{Clarity of Content}  & {\tiny Each element above is executed at a professional level so that someone can understand the goal, data, methods, results, and their validity and implications in a 5 minute reading at a cursory-level, and in a 30 minute meeting at a deep level (12 points). There is a clear connection from data to results to discussion and conclusion (12 points).} & \vspace{.2in}24              &  \vspace{.2in}18     \\ \hline
	\vspace{.2in}\textbf{Reproducibility}     & {\tiny Results are completely reproducible by someone with basic GIS training. There is no ambiguity in data flow or rationale for data operations. Every step is documented and justified.}                                                                                                                                                                     & \vspace{.2in}28              &     \vspace{.2in}24  \\ \hline
	\vspace{.2in}\textbf{Verification}        & {\tiny Results are correct in that they have been verified in comparison to some standard. The standard is clearly stated (10 points), the method of comparison is clearly stated (5 points), and the result of verification is clearly stated (5 points).}                                                                                                      & \vspace{.2in}20              &   \vspace{.2in}20    \\ \hline
	&                                                                                                                                                                                                                                                                                                                                                          & \vspace{.02in}100             &   \vspace{.02in}82     \\ \hline
\end{tabular}
\end{document}