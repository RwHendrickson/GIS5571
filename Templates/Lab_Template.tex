Lab Report

Title: <Delete this text in light grey throughout>
Notice: Dr. Bryan Runck
Author:
Date:

Project Repository: <if applicable weblink to public repository>
Google Drive Link: <if applicable with data, notebooks, etc.>
Time Spent: <report to the nearest quarter hour>

Abstract
<Delete this text in light grey throughout>
250 words max. Clearly summarize the following major sections. Each gets one or two sentences.  


Problem Statement
Describe the specific problem and the context. Provide an illustrative figure and/or context map here. In the table, translate the qualitative problem statement elements into specific requirements for the analysis.


Table 1. <insert caption>
#
Requirement
Defined As
(Spatial) Data
Attribute Data
Dataset
Preparation
1
Road network
Raw input dataset from MNDOT
Road geometry

Mn GeoSpatial Commons

2
High volume traffic
> 100 cars per hour

Volume
AADT Data

3






4









Input Data
Describe the data in two paragraphs max. Fill out the table.

Table 2. <insert caption>
#
Title
Purpose in Analysis
Link to Source
1
Minnesota Roads
Raw input dataset for routing analysis from MNDOT
Mn GeoSpatial Commons
2



3









Methods
Include a data flow diagram or screenshot from model builder. Do references in line (Rammankutty, 2033). Document any and all steps that you did to the input data in the data flow diagram. Provide natural language description of the most important steps, giving a narrative arc and provide well formatting screenshots with a boarder and centered throughout.

Resources on Data Flow Diagrams:
• https://www.visual-paradigm.com/tutorials/data-flow-diagram-dfd.jsp
• https://www.lucidchart.com/pages/data-flow-diagram/how-to-make-a-dfd

Figure 1. Data flow diagram. 

If appropriate, add in pseudo-code describing model algorithms and/or objects. If using mathematical equations, create a clear mapping between the reference equation, pseudo-code, and actual implementation in a programming language.

Results
Show the results in figures and maps. Describe how they address the problem statement. 

Follow best practice for map design, coloring, etc.




Results Verification
How do you know your results are correct? This can be a qualitative or quantitative verification.



Discussion and Conclusion
What did you learn? How does it relate to the main problem?




References
Use a common format


Self-score
Fill out this rubric for yourself and include it in your lab report. The same rubric will be used to generate a grade in proportion to the points assigned in the syllabus to the assignment.
Category
Description
Points Possible
Score
Structural Elements
All elements of a lab report are included (2 points each): 
Title, Notice: Dr. Bryan Runck, Author, Project Repository, Date, Abstract, Problem Statement, Input Data w/ tables, Methods w/ Data, Flow Diagrams, Results, Results Verification, Discussion and Conclusion, References in common format, Self-score
28

Clarity of Content
Each element above is executed at a professional level so that someone can understand the goal, data, methods, results, and their validity and implications in a 5 minute reading at a cursory-level, and in a 30 minute meeting at a deep level (12 points). There is a clear connection from data to results to discussion and conclusion (12 points).
24

Reproducibility
Results are completely reproducible by someone with basic GIS training. There is no ambiguity in data flow or rationale for data operations. Every step is documented and justified.
28

Verification
Results are correct in that they have been verified in comparison to some standard. The standard is clearly stated (10 points), the method of comparison is clearly stated (5 points), and the result of verification is clearly stated (5 points).
20



100

